\documentclass{beamer}
\usetheme{Madrid}
\mode<presentation>{}

\usepackage{amsfonts, amsmath, amssymb, bm}
\usepackage{caption}
\usepackage{tikz-cd}
\renewcommand{\r}{\mathbb{R}}
\newcommand{\ra}{\rightarrow}
\renewcommand{\hat}{\widehat}
\newcommand{\tract}{\mathcal{T}}
\newcommand{\confmet}{\bm{g}}

\usepackage{amssymb, amsthm,enumitem,amsmath,todonotes}

\newtheorem{theorem}{Theorem}[section]
\newtheorem{proposition}{Proposition}[section]
\newtheorem{lemma}{Lemma}[section]
\newtheorem{cor}{Corollary}[section]

\theoremstyle{definition}
\newtheorem{definition}{Definition}[section]
\newtheorem{remark}{Remark}[section]
\renewcommand{\epsilon}{\varepsilon}
\newcommand{\cupdot}{\mathbin{\mathaccent\cdot\cup}}
\newcommand{\n}{\mathbb{N}}
\newcommand{\z}{\mathbb{Z}}
\newcommand{\q}{\mathbb{Q}}
\renewcommand{\r}{\mathbb{R}}
\renewcommand{\c}{\mathbb{C}}
\newcommand{\seq}{\subset}
\newcommand{\norm}{\trianglelefteq}
\newcommand{\aut}{\textrm{Aut}}
\newcommand{\sym}{\textrm{Sym}}
\newcommand{\ra}{\rightarrow}
\newcommand{\im}{\textrm{im\,}}
\renewcommand{\ker}{\textrm{ker\,}}
\newcommand{\ex}{\backslash}
\renewcommand{\and}{\hspace{1em}\textrm{and}\hspace{1em}}

\newcommand{\tens}[3]{#1^{#2}{_{}} _{#3}}

\renewcommand{\sym}[1]{S^{#1}V^*}
\newcommand{\sk}[1]{\Lambda^{#1}V^*}
\newcommand{\st}{\,|\,}
\newcommand{\chris}[3]{\Gamma^{#1}{_{}}_{#2 #3} }
\newcommand{\udown}[3]{{#1}^{#2}{_{}}_{,#3}}

\newcommand{\set}[1]{\left\{ #1 \right\}}
\newcommand{\confmet}{\bm{g}}
\newcommand{\ric}{\textrm{Ric}}
\newcommand{\sca}{\textrm{Sc}}
\newcommand{\tract}{\mathcal{T}}
\renewcommand{\hat}[1]{\widehat{#1}}
\newcommand{\extd}{\textrm{d}}
\newcommand{\adjoint}{\mathcal{A} M}
\renewcommand{\phi}{\varphi}

\newcommand{\lpl}{
    \mbox{$
            \begin{picture}(12.7,8)(-.5,-1)
                \put(2,0.2){$+$}
                \put(6.2,2.8){\oval(8,8)[l]}
            \end{picture}$}}

\newcommand{\rpl}                         % +) or <+
{\mbox{$
            \begin{picture}(12.7,8)(-.5,-1)
                \put(0,0.2){$+$}
                \put(4.2,2.8){\oval(8,8)[r]}
            \end{picture}$}}

\newcommand{\lag}{\mathfrak{g}}
\newcommand{\lah}{\mathfrak{h}}

\newcommand{\gr}{\mathrm{gr}}

%Legacy commands; leaving these so that old versions of things will still compile. Try not to redefine these
\renewcommand{\n}{\mathbb{N}}
\renewcommand{\z}{\mathbb{Z}}
\renewcommand{\q}{\mathbb{Q}}
\renewcommand{\r}{\mathbb{R}}
\renewcommand{\c}{\mathbb{C}}

%Trying to swicth to a new convention
\newcommand{\NN}{\mathbb{N}}
\newcommand{\ZZ}{\mathbb{Z}}
\newcommand{\QQ}{\mathbb{Q}}
\newcommand{\RR}{\mathbb{R}}
\newcommand{\CC}{\mathbb{C}}
\newcommand{\id}{\mathrm{id}}

\newcommand{\EE}{\mathbb{E}}

\newcommand{\seq}{\subseteq}


\title[Distinguished curves]{Distinguished curves in projective and conformal geometry}
\author[Daniel Snell (UoA)]{Daniel Snell (University of Auckland)\\{ \bigskip\bigskip \small Joint work with
A.\ Rod Gover and Armann Taghavi-Chabert}}
\date{\today}

\begin{document}

\begin{frame}
  \titlepage
\end{frame}

\section{Introduction}

\begin{frame}{Geometry according to Klein}
  \vspace{-10pt}
  \begin{columns}
    \begin{column}{0.6\textwidth}
  One of the earliest attempts to describe what is meant by ``a geometry'' is due
  to Felix Klein.
  \end{column}
    \begin{column}{0.4\textwidth}
      \begin{figure}
        \includegraphics[width=0.55\textwidth]{klein.jpg}
        %\caption*{Felix Klein}
      \end{figure}
    \end{column}
  \end{columns}
  %\vspace{-20pt}
  One may describe such a geometry as either
  \begin{itemize}
    \pause
    \item a pair $(G,P)$, where $G$ is a Lie group and $P\subset G$ is a closed
      subgroup, such that the coset space $G/P$ is connected; or
    \pause
    \item a Lie group $G$ which has the structure of a smooth principal
      $P$-bundle over the coset space $G/P$:
      \[
        P \to G \to G/P.
      \]
  \end{itemize}
  \pause
  Many of the simplest geometries are realized this way, for example,
  $\mathbb{E}^n$, $\mathbb{S}^n$ and $\mathbb{H}^n$. 

  However, all the geometries arising this way are flat. 

  So how do we incorporate curvature? 
\end{frame}

\begin{frame}{Geometry according to Cartan}

  This was accomplished by Cartan:

\[
\begin{tikzcd}[ampersand replacement=\&, row sep=huge, column sep=huge]
\textnormal{Euclidean geometry} \arrow[r] \arrow[d] \& \textnormal{Riemannian geometry} \arrow[d]\\
\textnormal{Homogeneous spaces} \arrow[r] \& \textnormal{Cartan geometry}
\end{tikzcd}
\]

\pause
Cartan geometries simultaneously generalize both of these, giving curved versions of homogeneous spaces which locally look like the flat \( G/P \).

Cartan's idea was to replace the group $G$ with a principal $P$-bundle, \(
\mathcal{G} \), and to include an object called the \emph{Cartan connection},
$\omega$, which is a weakened Maurer-Cartan form. 

The curvature of this Cartan connection is then defined by the structure
equation:
\[
  \Omega := \mathrm{d} \omega + \dfrac{1}{2} [\omega, \omega].
\]

\end{frame}

\begin{frame}{Conformal and projective geometry}

  \begin{definition}[Conformal manifold]
    A conformal manifold is a pair \( (M, \bm{c}) \), where \( \bm{c} \) is an equivalence class of Riemannian metrics, with \( \hat{g} \sim g \) if, and only if, \( \hat{g} = \Omega^2 g \) for some positive function \( \Omega \in C^\infty (M) \).
  \end{definition}

  \begin{definition}[Projective manifold]
    A projective manifold is a pair \( (M, \bm{p}) \), where \( \bm{p} \) is an
    equivalence class of projectively related affine connections. Two connections \( \nabla \), \(
    \tilde{\nabla} \) are said to be \emph{projectively related} if, and only if, they have the same geodesics
    as unparametrized curves.
  \end{definition}

  \vspace{-1em}
  \begin{columns}
    \begin{column}{0.6\textwidth}
      \pause
      Conformal and projective manifolds are examples of 
      a particular subclass of Cartan geometries called \emph{parabolic geometries}, where \( G \) is semisimple, and \( P \subset G \) is a \emph{parabolic} subgroup.
    \end{column}
      \begin{column}{0.4\textwidth}
        \hspace{1.5em}
        \begin{figure}
          \includegraphics[width=0.35\textwidth]{cartan.jpg}
          \caption*{\'{E}lie Cartan}
        \end{figure}
      \end{column}
  \end{columns}

\end{frame}

\begin{frame}{History}

Conformal and projective geometry have been long studied, both for their own
sake and for their relations to other areas of mathematics and physics.
\vspace{1em}\\
\pause
The study of these geometries is difficult since they do not possess distinguished connections on their tangent bundles.
\vspace{1em}\\
\pause
Work of Bailey, Eastwood and Gover (1994) provides an elegant solution: work on a vector bundle of slightly higher rank. 

These are the \emph{tractor bundles}.

\end{frame}

\begin{frame}{Tractor bundles}
  \begin{definition}[Conformal tractor bundle]
    Let \( (M, \bm{c}) \) be a conformal manifold.
    The (standard) \emph{conformal tractor bundle}, \( \mathcal{T} \), fits into an exact sequence
    \pause
    \[
      0 \to \mathcal{E}[-1] \overset{X}{\to} \mathcal{T} \to J^1 \mathcal{E}[1] \to 0.
    \]
    \pause
    The invariant bundle map \( X \), called the \emph{canonical} or \emph{position} tractor will be of central importance to our main results.
    Under a choice of metric \( g \in \bm{c} \), one has an isomorphism 
    \[
      \mathcal{T} \overset{g}{\cong} \mathcal{E}[1] \oplus \mathcal{E}^a [1] \oplus \mathcal{E}[-1].
    \]
  \end{definition}

\end{frame}

\begin{frame}{Tractor bundles (cont.)}
  The conformal tractor bundle also comes equipped with an invariant connection and an invariant metric which is compatible with this connection.
  \pause
  \begin{Theorem}[Bailey, Eastwood, Gover 1994]
    Let \( (M, \bm{c}) \) be a conformal manifold with tractor bundle \( \mathcal{T} \).
    Then there is a conformally invariant connection on \( \mathcal{T} \), defined in a choice of metric by
    \pause
    \[
      \nabla_a^{\mathcal{T}}
      \begin{pmatrix}
        \sigma \\ 
        \mu_b \\ 
        \rho
      \end{pmatrix}
      \overset{g}{:= }
      \begin{pmatrix}
        \nabla_a\sigma - \mu_a \\ 
        \nabla_a\mu_b + \confmet_{ab} \rho + \mathsf{P}_{ab} \sigma \\ 
        \nabla_a \rho - \mathsf{P}_{ab} \mu^b
      \end{pmatrix}.
    \]
    \pause
    The invariant metric can be written as 
    \[
      h^{AB} = 
      \begin{pmatrix}
      0 & 0 & 1 \\
      0 & \confmet_{ab} & 0\\
      1 & 0 & 0\\
      \end{pmatrix}.
    \]
  \end{Theorem}
\end{frame}

\begin{frame}{Distinguished curves}
  Parabolic geometries also possess a notion of distinguished curves.
  We will focus on two cases: \\
  \begin{center}
    \begin{tabular}{ | c | c | }
      \hline
      Geometry & Distinguished curves \\ \hline
      Projective & \phantom{(Usual) geodesics} \\ \hline 
      Conformal & \phantom{Null geodesics and conformal circles} \\
      \hline
    \end{tabular}
  \end{center}
  \phantom{For these different types of curves, we get nice characterizations of the distinguished curves using tractor calculus methods.}
\end{frame}

\begin{frame}{Distinguished curves}
  Parabolic geometries also possess a notion of distinguished curves.
  We will focus on two cases: \\
  \begin{center}
    \begin{tabular}{ | c | c | }
      \hline
      Geometry & Distinguished curves \\ \hline
      Projective & (Usual) geodesics \\ \hline 
      Conformal & \phantom{Null geodesics and conformal circles} \\
      \hline
    \end{tabular}
  \end{center}
  \phantom{For these different types of curves, we get nice characterizations of the distinguished curves using tractor calculus methods.}
\end{frame}

\begin{frame}{Distinguished curves}
  Parabolic geometries also possess a notion of distinguished curves.
  We will focus on two cases: \\
  \begin{center}
    \begin{tabular}{ | c | c | }
      \hline
      Geometry & Distinguished curves \\ \hline
      Projective & (Usual) geodesics \\ \hline 
      Conformal & Null geodesics and conformal circles \\
      \hline
    \end{tabular}
  \end{center}
  \pause 
  For these different types of curves, we get nice characterizations of the distinguished curves using tractor calculus methods.
\end{frame}

\section{Main theorems}

\begin{frame}{Main theorem for projective geometry}
  \begin{Theorem}[Gover, S., Taghavi-Chabert 2018]
    Let \( (M,\bm{p}) \) be a projective manifold, and \( \gamma \) a curve in \(
    M \).
    Then \( \gamma \) is an unparametrised oriented geodesic if, and only if,
    along \( \gamma \) there is a non-zero parallel 2-tractor \( \Sigma^{A B}\in
    \Gamma\left( \mathcal{E}^{ [A B] }\right) \) such that
    \[
      X \wedge \Sigma = 0.
    \]
  \end{Theorem}
  \pause
  The theorem consists of two conditions:\\
  \begin{enumerate}
    \pause
    \item \( X \wedge \Sigma = 0\) is a kind of incidence relation, and
    \pause
    \item the condition \( \nabla_{\dot{\gamma}} \Sigma = 0 \) exactly recovers the geodesic equation.
  \end{enumerate}
\end{frame}

\begin{frame}{Main theorem for conformal geometry}
  The geodesic equation is not conformally invariant; instead one considers the
  \emph{conformal circles}.
  These admit a similar characterisation to geodesics in projective geometry.
  \pause
  \begin{Theorem}[Gover, S., Taghavi-Chabert 2018]
    Let \( (M,\bm{c}) \) be a conformal manifold.
    A nowhere-null curve \( \gamma \) is an oriented conformal circle if, and
    only if, along \( \gamma \) there is a non-zero parallel 3-tractor \(
    \Sigma^{ABC} \in \Gamma(\mathcal{E}^{[ABC]}) \) such that
    \[
      X \wedge \Sigma = 0.
    \]
  \end{Theorem}
  \pause
  Again the result consists of an incidence relation and an equation exactly
  equivalent to the appropriate distinguished curve equation.
  \pause
  \hspace{1em}\\
  NB: The theorem for null geodesics is almost identical to the projective geometry version.
\end{frame}

\section{Conserved quantities in projective and conformal geometry}

\begin{frame}{Application: conserved quantities}
  One of the most important results about geodesics of a Riemannian manifold is
  the following classical theorem.
  \pause
  \begin{Theorem}
    Let \( (M,g) \) be a Riemannian manifold.
    Let \( k \) a Killing vector field for the metric \( g \), and \( \gamma \)
    a geodesic for the Levi-Civita connection \( \nabla \).
    Then the scalar-valued function \( g(\dot{\gamma},k) \) is conserved along
    \( \gamma \), i.e.
    \[
      \nabla_{\dot{\gamma}} g(\dot{\gamma}, k) = 0.
    \]
  \end{Theorem}
  \pause
  Conserved quantities are useful in numerous applications, including
  physics, superintegrable systems, and separation of variables in the study of
  differential equations.

  It is therefore natural to ask about quantities which are conserved on a conformal or projective manifold. 
\end{frame}

\begin{frame}{Towards conserved quantities}
  \begin{block}{A na\"{i}ve approach}
    We take inspiration from the classical result. 
      This required two things:
      \begin{itemize}
        \pause
        \item a distinguished curve, and
        \pause
        \item a \emph{symmetry} of the geometry.
      \end{itemize}
      \pause
      We have already a tractor which describes the distinguished curves. Therefore we would like to find a section \( S \) of some tractor bundle such that e.g. \( \nabla_{\dot{\gamma}} \left(\Sigma \cdot S \right) = 0 \). 
      In light of our earlier theorems, this is equivalent to the simpler
      \[
        \Sigma \cdot \nabla_{\dot{\gamma}} S = 0.
      \]
      This tractor \( S \) should ideally be a solution to some geometric PDE on the manifold...
  \end{block}
\end{frame}

\begin{frame}{A minor digression: elements of BGG theory}
  Let \( \mathcal{V} \) be a tractor bundle.
  Then one may form the \emph{twisted de Rham} sequence.
  We shall mainly be interested in the first square:
  \pause
  \[
    \begin{tikzcd}[ampersand replacement=\&, row sep=huge, column sep=huge]
      \mathcal{V} \arrow[r, "\nabla^{\mathcal{T}}"] \arrow[d, "\Pi_0"] \& T^*M \otimes \mathcal{V} \arrow[l, "\hspace{-2em}\partial^*", near start, bend left=20] \arrow[d, "\Pi_1"]\\
      \mathcal{H}_0 \arrow[r, "\Theta"] \arrow[u, dashed, "L", bend left=50] \& \mathcal{H}_1
    \end{tikzcd}
  \]
  where \( \Theta := \Pi_1 \circ \nabla^{\mathcal{T}} \circ L \).
  \pause
  \begin{block}{}
  One may ask for solutions to the equation \( \Theta \sigma = 0 \) for various tractor bundles \( \mathcal{V} \).
  This class of \emph{BGG equations} turns out to include a large number of interesting geometric PDEs, for example: 
  \begin{itemize}
    \item the Killing (form/tensor) Equation, 
    \item the Metrisability Equation,
    \item the Almost-Einstein Equation...
  \end{itemize}
  \end{block}
\end{frame}

\begin{frame}{Putting it all together}
  \hspace{1em}
  \begin{block}{Key idea}
    The splitting operator \( L \) allows us to convert solutions to geometrically interesting PDEs into sections of certain tractor bundles.
    We therefore might try replacing \( S \) in the above with \( L(\sigma) \), where \( \sigma \) solves some BGG equation.
  \end{block}
  \hspace{1em}
  \pause

  The requirement that 
    \[
      \Sigma \cdot \nabla_{\dot{\gamma}} L(\sigma) = 0
    \]
    is not too onerous, as demonstrated by the existence of \emph{normal solutions} which one has in e.g. the flat models.
\end{frame}

\begin{frame}{A conformal example}
  \begin{Theorem}
    Let \( (M,\bm{c}) \) be a conformal manifold.
    Suppose \( \gamma \) is a conformal circle, and \( k_{ab} \) is a conformal
    Killing 2-form.
    The form \( k_{ab} \) corresponds to a cotractor 3-form \( \mathbb{K}_{ABC}
    \).
    Then the scalar \( \Sigma^{ABC} \mathbb{K}_{ABC} \) is conserved along \(
    \gamma \).
    In a scale, one has
    \[
      \Sigma^{ABC} \mathbb{K}_{ABC} = \mathbf{u}^a \mathbf{a}^b k_{ab} \mp \frac{1}{n-1} \mathbf{u}^a \nabla^c k_{ca}.
    \]
  \end{Theorem}
  \pause
  \begin{proof}[Proof (sketch)]
    % Applying the Leibniz rule for the tractor connection, we calculate
    % \pause
    % \begin{align*}
    %   u^a \nabla^\mathcal{T}_a \left( \Sigma^{ABC} \mathbb{K}_{ABC} \right) &=
    %   \left( u^a \nabla^\mathcal{T}_a \Sigma^{ABC} \right) \mathbb{K}_{ABC} +
    %   \Sigma^{ABC} u^a \nabla^\mathcal{T}_a \mathbb{K}_{ABC}\\
    %   &= \Sigma^{ABC} u^a \nabla^\mathcal{T}_a \mathbb{K}_{ABC},
    % \end{align*}
    % since according to our main result, \( \Sigma^{ABC} \) has derivative zero
    % along \( \gamma \).
    % \pause
    % Using explicit formulae for \( \Sigma \) and \( \mathbb{K} \), one sees
    % that this remaining term also vanishes.
    Two different ways:
    \begin{itemize}
      \item use the explicit formula for the quantity given above;
      \item use techniques from BGG theory, work of \v{C}ap, Hammerl, Sou\v{c}ek, Somberg; and Gover, \v{S}ilhan...
    \end{itemize}
  \end{proof}
\end{frame}

\begin{frame}{More examples}
  During our work, we calculated explicitly several other examples, using
  solutions to different BGG equations.\\
  \begin{itemize}
    \pause
    \item In the conformal case:
      \[
      S^{AB} H_{AB},
      \]
      where \( S^{AB} \) is constructed from \( \Sigma^{ABC} \) and
      \( H_{AB} \) is the tractor corresponding to the solution to a certain
      conformally-invariant third-order PDE.
    \pause
    \item And in the projective case:
      \[
        \Sigma^{A_1 B_1} \Sigma^{A_2 B_2} H_{A_1
        A_2} H_{B_1 B_2},
      \]
      where \( H_{A B} \) corresponds to the projective version of the
      aforementioned third-order equation.\\
      \pause
    \item Finally, in projective geometry:
      \[
        \Sigma^{A B} \mathbb{K}_{A B},
      \]
      where \( \mathbb{K}_{A B} \) comes from a Killing vector field recovers the well-known conserved quantity.
  \end{itemize}
\end{frame}

\section{Ongoing work}

\begin{frame}{Ongoing work}
  \begin{block}{}
    We are currently working on developing characterizations of the distinguished curves from some other parabolic geometries, as well as fitting our results into the existing theory of distinguished curves in parabolic geometries. \\
    \hspace{1em}\\
    \pause
    Ultimately this machinery along with the conserved quantities it produces should have applications to integrable systems and separability of PDE.
  \end{block}
\end{frame}

\begin{frame}{Thank you}
  \begin{block}{}
  Details in \href{https://arxiv.org/abs/1806.09830}{arXiv:1806.0983}\\
  \hspace{1em}\\
  Slides: \href{https://github.com/snelltrail/anzamp_2019}{https://github.com/snelltrail/austms\_2018}\\
  \hspace{1em}\\
  Contact: \texttt{\href{mailto:daniel.snell@auckland.ac.nz}{daniel.snell@auckland.ac.nz}}
  \end{block}
\end{frame}

\end{document}


